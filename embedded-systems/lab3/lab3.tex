\documentclass[12pt]{article}
\usepackage[utf8]{inputenc}
\usepackage[T1]{fontenc}
\usepackage[polish]{babel}
\usepackage{graphicx}
\usepackage{listings}
\usepackage{color}
\usepackage{hyperref}
\usepackage{amsmath}
\usepackage{url}
%\usepackage{fontspec}

\definecolor{codegreen}{rgb}{0,0.6,0}
\definecolor{codegray}{rgb}{0.5,0.5,0.5}
\definecolor{codepurple}{rgb}{0.58,0,0.82}
\definecolor{backcolour}{rgb}{0.95,0.95,0.92}
%\setmonofont{Consolas}

\lstdefinestyle{mystyle}{
    backgroundcolor=\color{backcolour},   
    commentstyle=\color{codegreen},
    keywordstyle=\color{magenta},
    numberstyle=\tiny\color{codegray},
    stringstyle=\color{codepurple},
    breakatwhitespace=false,         
    breaklines=true,                 
    captionpos=b,
	basicstyle=\ttfamily\footnotesize,                    
    keepspaces=true,                 
    numbers=left,                    
    numbersep=5pt,                  
    showspaces=false,                
    showstringspaces=false,
    showtabs=false,                  
    tabsize=2
}
 
\lstset{style=mystyle}

\usepackage{fancyhdr}
\pagestyle{fancy}
%\fancyhead[L]{Systemy Wbudowane}
\fancyhead[L]{Komunikacja bezprzewodowa w podczerwieni\\ standard RC5}
\fancyhead[R]{Laboratorium 3}

\author{Kamil Wójcik}
\title{Sprawozdanie}
\begin{document}
\begin{titlepage}
 
\begin{center}

{\Large {Wydział Fizyki, Matematyki i Informatyki\\Politechnika Krakowska
} }
\\[1cm] 
\includegraphics[scale=0.4]{pk.png}
\\[1cm]

{\Huge \textsc{Instytut Teleinformatyki}}\\[0.7cm]
{\Huge Systemy Wbudowane}\\[2cm]
{\Large Sprawozdanie z Laboratorium 3}\\[2cm]

\begin{minipage}[l]{0.5\textwidth}
    \begin{flushleft}
        \textbf{\textsf{Prowadzący:}}\\
        \large dr hab. Zbisław Tabor\\
        \large prof. dr hab. inż. Piotr Malecki\\
        \large mgr inż. Paweł Pławiak 
        \end{flushleft}
\end{minipage}
\begin{minipage}[l]{0.3\textwidth}
    \begin{flushright}
        \textbf{\textsf{Autor:}}\\
        \linespread{1}
        \large Kamil Wójcik
    \end{flushright}
\end{minipage}
 
\end{center}
 
\end{titlepage}



\section{Wprowadzenie}
Celem laboratorium było zapoznanie się z systemem przerwań w układach mikrokontrolerów, na przykładzie zestawu ZL15AVR wyposażonego w AVR ATmega32.

\section{Zadanie}
Celem wybranego zadania(2.2) było napisanie programu odliczającego sekundy na wyświetlaczu 7-segmentowym wykorzystując przerwania od Timera0.

\subsection{Obsługa przerwań}
Zgodnie z wymaganiem do odliczania czasu użyjemy przerwań od Timera0. Zastosowany zestaw uruchomieniowy posiada wbudowany oscylator o $f=1MHz$. Zakładając najwyższą wartość preskalera $T = 1024$ oraz fakt że użyjemy przerwań od przepełnienia rejestru możemy określić że potrzebna liczna przerwań do odliczenia $1s$ będzie wynosić $\approx$ 3.81. Aby rozwiązać ten problem wprowadzimy dodatkową zmienną która będzie określała liczbę przerwań.

\[C = \frac{\frac{f_{clk}}{f_{czas}}}{R T}\]

gdzie:\\
$C$ - licznik przerwań\\
$f_{clk}$ - częstotliwość pracy oscylatora (1MHz)\\
$f_{czas}$ - częstotliwość pożądana (1Hz)\\
$R$ - ilość cykli timera do przerwania (255). W naszym przypadku przerwania ustawione są od przepełnienia a Timer0 jest timerem 8-bitowym\\
$T$ - wartość preskalera(64)\\
\\[0.5cm]
Podstawiając do wzoru otrzymujemy: $C = 61,2745$.
Łatwo można obliczyć że czas odliczany tym sposobem będzie wynosił $\approx0.9955s$. Na takiej precyzji poprzestaniemy. Wymienię tylko sposoby jakimi można osiągnąć większą precyzję.
\begin{itemize}
\item Wprowadzenie kolejnej zmiennej pozwalającej dokładniej zliczać przerwania przy mniejszej wartości preskalera
\item Zastosowanie zegara czasu rzeczywistego
\end{itemize}

W bloku obsługi przerwania pomniejszamy licznik przerwań. W przypadku gdy limit został osiągnięty wyświetlana cyfra zostaje zwiększona a w razie przepełnienia wyzerowana. Po zwiększeniu cyfry licznik przerwań jest resetowany. 

\subsection{Obsługa wyświetlacza}
Do sterowania wyświetlaczem posłużymy się portem A. W tym celu ustawiamy wartość rejestru DDRA w stan wysoki(0xFF). W celu konwersji wartości cyfr do wartości sterującej zapalonymi segmentami wyświetlacza definiujemy tablice \textit{digitValues} jej wartości ustawiamy w funkcji \textit{init}. Przy takim założeniu piny segmentów powinny być podłączone w kolejności \textit{gfedcba} począwszy od pierwszego pinu Portu A. Do ostatniego pinu portu A podłączamy pin wyboru cyfry wyświetlacza oraz ustawiamy wyjście w stan wysoki (0x80).\\

\lstinputlisting[language=C, caption=Program zadanie 2.2]{listing1.c}

\begin{thebibliography}{99}
\bibitem{skr} Instrukcja do ćwiczeń laboratoryjnych [\url{http://iti.pk.edu.pl/index.php/pl/dydaktyka/materia%C5%82y-dydaktyczne/sw.html}]
%url = {http://iti.pk.edu.pl/attachments/article/55/Lab_2_instrukcja.pdf}
\bibitem{} Nota katalogowa mikrokontrolera ATmega8 [\url{http://www.atmel.com/Images/Atmel-2486-8-bit-AVR-microcontroller-ATmega8_L_datasheet.pdf}]
\bibitem{} Nota katalogowa odbiornika podczerwieni TSOP31236 [\url{http://www.vishay.com/docs/82492/tsop312.pdf}]
\end{thebibliography}
\end{document}
